\chapter{Using GAMBIT GUI}

\begin{figure}
$$\image{3cm;3cm}{biggamb}$$
\caption{Gambit GUI showing normal and extensive
forms}\label{fig:biggamb}
\end{figure}

The Gambit GUI consists of two self-contained modules for representing,
displaying and manipulating games.  A game can be viewed in either the
extensive or normal forms.  Although independent, the modules are
seamlessly integrated and it is possible to solve the game in the normal
form while viewing it in the extensive.  It is also possible to go from
the extensive form representation to the normal form (but not vice versa
at the present).   You can select the module to be used in the File menu,
in one of two ways.

\begin{itemize}
\item
Build a new game by selecting File, New , and then choosing either
Extensive or Normal.
\item
Load an existing game by selecting File, Open, and then selecting a file
that contains a previously saved game in either extensive or normal form.
\end{itemize}

\section{File menu}
When you save an extensive or normal form game in Gambit, it will be
written in a standard format..  Files containing extensive form games are
given the default extension of \verb+.efg+, and files containing normal
form games are given the default extension \verb+.nfg+ .  When you select
File Open, you will get a file loading dialog box from which you can
either directly enter the name of the file, or browse the directory system
for files with the extensions \verb+.efg+ or \verb+.nfg+.  If you select a
normal form file, you will bring up the normal form module of the gui.  If
you select an extensive form file, you will bring up the extensive form
module of the gui.

\section{Data Types}
Gambit allows you to do computations in either floating point (double) or
exact arithmetic (rational).  You must select the data type whenever you
load a game from an external file or create a new game.  If you want to
change to a different data type, save the file, and reload it as a
different data type.  However, note that saving a rational file and
reloading it as double could lose information.  (For example, 1/3 wiould
be converted to its nearest decimal representation.)

\begin{description}
\item[Double:]  If you select Double, the game will be represented in double precision floating point numbers, which on most machines result in about 15 places of accuracy.  The benefit of floating point calculations is speed, since floating point numbers are designed to fit in a fixed amount of storage, and arithmetic operations are coded in hardware.  However,  in some games, floating point calculations can result in roundoff errors that will lead to either incorrect solutions, or failure to find solutions that exist.  

\item[Rational:]  If you select Rational, the game will be represented in rational numbers, which are represented internally as the quotient of two arbitrary precision integers.  Doing calculations in rationals will guarantee that the answer is exact, at least for those algorithms that support rational calculations.  However calculations in rationals are slower than double calculations, by one or two orders of magnitude.  Currently the only algorithms supporting calculations in rationals are the following two person algorithms: LpSolve, LcpSolve, and EnumPure and EnumMixed.  
\end{description}

\section{Normal Form GUI}

\begin{figure}
$$\image{0cm;0cm}{nfg}$$
\caption{A 3 player normal form}\label{fig:nfg}
\end{figure}

In the normal form representation, the game is viewed as a 2-dimensional
window into an N-dimensional matrix.  Each 2D window shows a table of
payoffs for each player as a function of the strategies for two of the
players, holding the strategies of all of the other players (if there are
any) fixed.  Each cell contains the payoff vector with one entry per
player for the strategy profile determined by this cell.  The profile
itself is set by using a combination of settings for the row and column
players plus the strategy settings for the rest of the players.  {\em The
payoff vector can be edited by double-clicking on the cell and entering
new values in the dialog.}

\subsection{Row and Column Players} The row and column player choice boxes
determine which players' strategies get displayed on the horizontal and
vertial axis of the matrix.  Each box can contain a value from 1 to the
total number of players in the game.  The dimensions of the matrix are
determined by the number of strategies for the row and column players.
Note that it is meaningless to select the same player for both the row and
the column.  If the game has more than two players, a warning will be
ussued and no action will be taken.  In the case of a two player game, the
row and the column players will be switched.

\subsection{Strategy Profile}
The array of choice boxes labeled "Profile" at all times reflects the
strategies picked by each player to achieve the payoffs shown in the
highlighted cell.  The n'th choice box can contain a value from 1 to the
total number of strategies the n'th player has.  When one of the choice
boxes is changed, one of two things can happen:
\begin{enumerate}
\item If the choice box number is not equal to either the row or
the column player, the entire matrix will be updated with new values to
reflect the new 2D view into the matrix.
\item If the choice box number was either the row or the column player,
the highlighted cell will move to reflect the new strategy.
\end{enumerate}

\subsection{Normal Form GUI Display}
In order to accomodate as many different platforms and tastes as possible,
the display features of the normal form matrix are highly configurable.  A
large number of these configuration features are common to all the 'table'
displays in Gambit.  Refer to \helprefn{Table Window}{TableWindow} section
of this manual for a detailed explanation of those features.  Another set
of display options can be accesed through the Display->Features menu.

\subsubsection{Normal Form Features}
The features dialog allows the choice to display or not to display extra
normal form data.  This data is obtained from running various solution
algorithms.  Three data sets may be available depending on the solutions
run on the game.  Any of them can be turned off through this dialog.
\begin{enumerate}
\item Strategy probability data.  Many normal form solution
algorithms return solutions in the form of mixed or pure strategy
equilibria.  If such an algorithm was run, an extra row and column will be
added to the matrix to show the calculated probability of row and column
players choosing the respective strategies.  The value in the lower left
cell is the probability of players other than row and column choosing
their strategies to achieve this profile.
\item Value to player data.  This data is dependent on the previous item
and is just the sum of the products of the payoffs to row and column
players times the probability of them choosing each strategy.
\item Strategy dominance data.  This data is generated by running
the Eliminate Dominant strategies algorithm.  For each strategy of row and
column players, it will show by which, if any, strategy, that strategy is
dominated.  For more information see the discussion of the ElimDom
algorithm.
\end{enumerate}

\subsection{Normal Form Solutions}\label{NormalFormSolutions}
All of the available solution algorithms are accessed through the
Solve->Solve menu.  This dialog provides for both the creation of new
solutions and inspection of already existing ones.  If a particular
\helpref{algorithm}{SolutionAlgorithms}  is not aplicable for the current game, 
its selection will be
disabled.  If there already exist solutions created with the selected
algorithm, the {\em Look} button will become enabled.  Pressing this
button will take you to a solutions inspection window.  That window
depends on the type of the algorithm that is selected.  For algorithms
that generate mixed strategy equilibria, the  \helpref{mixed
solutions}{NormalSolutionInspect} window will appear.  For the elimination
of dominated strategies, the
\helpref{elimdom}{ElimDomInspect} window will appear.  No
inspection windows exist for algorithms that generate pxi type files.

If the current game was generated from an extensive form game, and if no
changes have been made to either game, there will exist a link between the
two forms.  In this case, the {\em Extensive form} checkbox will be
enabled.  Checking this box will cause all generated solutions (if
applicable to the solution type) to be converted back into behavioural
profiles and projected back to the extensive form window.  The solutions
can then be examined in the extensive form display.


\subsubsection{Normal Solutions Inspect}\label{NormalSolutionInspect}
\begin{figure}
$$\image{3cm;3cm}{nfgsoln}$$
\caption{Normal Form Solution Inspect Window}\label{fig:nfgsoln}
\end{figure}


The solution inspection window is a very powerful and complex part of the GUI.
The basic functionality consists of displaying the MixedProfiles that contain the
probability data for the solutions.  The probabilities are arranged with each 
player's strategies on a separate line, and strategies in consequtive cells.  
Each cell consists of the strategy number (or name), followed by a ':' and followed
by the probability of that strategy.
If the {\em display zero prob} option is not turned on, strategies with zero probability
will not be displayed. 

The value of the solution to each player can be displayed by selecting the 
{\em Display Equilibrium Values} option.

The number of the solution that is currently selected will be hilighted in the first
column.  To change the current solution (and thus change the display in the 
corresponding NF window), double click (control-click in unix) on the \# of the
desired solution.  To quickly browse the solutions, the {\em Update Solutions Dynamically}
option may be used.  With this option on, the solution will change automatically once
the cursor moves to a different solution \#.  To remove solution display from the NF window,
double click on the first row.

If the underlying NF was generated from an EF, and if the link between the two is still
valid, the {\em NF->EF} button will be enabled.  To project the solution to the EF and
immediately display it there, press this button.

Note that multiple solution inspection windows can exist at one time.  Filtering will
be eventually implemented to allow the selection of a particular type of solution to
be displayed.

Note that all solution windows will be deleted if any changes are made to the underlying
NF.

\normalbox{For configuration and output features of this dialog see
the generic \helprefn{Table Window}{TableWindow} description.}


\subsubsection{Examining NF Supports}\label{ElimDomInspect}
For a detailed explanation of a {\em Support} see the appropriate theory section.
Supports can be generated either by the elimination of strategies algorithm,
or explicitly by pressing the \helpref{New Support}{NewSupport} button in this dialog.  Two supports
are defined for every NF window:
\begin{enumerate}
\item The displayed support determines the dimensions of the NF that is actually
displayed in the window.
\item The current support determines what support will be used for all the solution
computation and elimination of dominated strategies. 
\end{enumerate}
By default, these supports are identical.  However, and advanced user may wish to
display one support while working on another.  Solution display may become 
confusing (strat probs do not add up to 1) in this case.

The dimensionality of the supports is displayed in the text box above the selection
choicebox.

\subsubsection{Creating NF Supports}\label{NewSupport}
A new supports starts with the default of the {\em full support}.  That is, for each player,
all the strategies are included in the support.  To deselect a player's strategy, click on it
in the appropriate listbox.  The strategy will no longer be hilighted.  At least one strategy
must remain selected in each listbox.  Upon returning to the \helpref{examination}{ElimDomInspect} dialog, the newly created support will be added
to the support list.



\subsection{Default Accelerator Keys}\label{NormFormDefAccl}
None


%---------------------------------- EFG -------------------------
\section{Extensive Form GUI}

\begin{figure}
$$\image{0cm;0cm}{efg}$$
\caption{A 3 player extensive form}\label{fig:efg}
\end{figure}

In the extensive form, the game is represented as a topological tree.  See
the Extensive Form section for a detailed explantion of this form.
Compared to the Normal Form, the Extensive Form interface is much richer
and thus considerably more complex.  A major portion of the functionality
is devoted to the tree building.  Another set of functions deals with
customizing the display, and yet another set of functions takes care of
the solutions and their display.

\subsection{Tree Building Functions}
This section assumes a working knowledge of the conventions used in the
GAMBIT extensive form display (discussed in XXXXXXXXX).

\subsubsection{Node Menu}
\begin{itemize}
\item Add and Insert node.  This is the most commonly used tree building
function.  Each node is determined by the player it belongs to and the
infoset it belongs to.  There are two ways to add a node: either by
choosing the number of branches the node is to have and thus create a new
infoset, or by choosing an existing infoset this node will belong to.  To
add a node you must first select a player.  The player can be either
chance or an existing player, or a "New Player."  If New Player is chosen,
a new player will be created and given a default name "Player \#." After
selecting the player, you must decide which of the two methods described
above is to be used.  If a new infoset is to be created, just enter the
number of branches desired.  If this node is to belong to an existing
infoset, choose the desired infoset from the Iset choicebox.  If the node
created was a CHANCE node, you will then be prompted for the probabilities
associated with each branch.
\item Delete Node.  When a node is deleted, one of its children (if any
exist) will replace it.  The other children will be destroyed.  Note that
you can not delete the ROOT node.
\item Label Node.  Each node can have a label associated with it. 
This function allows the entering or modification of this label.  The
display of these labels is controled in the Legends Dialog.
\item Probs.  This option is only valid when the cursor is on a
CHANCE node.  It allows the explicit setting of each branch probability.
\item Player.  This allows changing the player that has the choice
at this node.  You can choose any player except the one that is currently
selected.
\item Set and GoTo Mark.  The GUI allows to mark (memorize) one
node for later use.  This can be useful for example in a very large game
to quickly move from one part of the game to another.  The marked node is
also required in many tree operations.
\end{itemize}

\subsubsection{Branch Menu}
\begin{itemize}
\item Insert and Delete branch.  This option allows changing the
number of branches at this node.  Any changes made here will be relfected
on all nodes that belong to the same infoset.
\item Label branch.  Each action in an infoset can have a label. 
This allows the setting or changing of this label.
\end{itemize}

\subsubsection{Tree Menu}
\begin{description}
\item[Copy] copy
\item[Move] move
\item[Delete]  This node and all its children will be deleted.
\item[Label]  The label pertains to the game as a whole and will be
displayed on the titlebar of the window.
\item[Players]  Each player must have a name.  Although not required, it
is highly recommended that these names be uniqe.  When a player is first
created, it is given a default name "Player \#."  This dialog allows you to
change these default names to something appropriate to the game.
\item[Outcomes]  See the \helprefn{Outcomes section}{OutcomesGUI}
for a detailed explanation of this dialog.
\end{description}

\subsubsection{Infoset Menu}

\subsubsection{Display Menu}

\subsection{Command language logging}
GAMBIT can also be used without the GUI, by running the command line
version of the program.  However, it is often convinient to build an
extensive form tree in the GUI, and then operate on it in the command
language.  To facilitate this, the GAMBIT GUI can log all the commands
executed through it to a file in the command language format.  To start
logging, select the {\em File->Logging} menu.  All the commands executed
following this will be saved to a new file or appended to an existing one.
To stop logging, select the {\em File->Logging} menu again.  The resulting
file can now be loaded into the gcl (gambit command line) to produce
results identical to those achieved in the gui.


\subsection{Outcomes GUI}\label{OutcomesGUI}
\begin{figure}
$$\image{0cm;0cm}{outcomes}$$
\caption{A typical outcomes window}\label{fig:outcomes}
\end{figure}

The outcomes dialog arranges the payoff data for the game in a matrix
form.  Each row contains an entire outcome vector with one entry per
player.  If an entry is blank, it is assumed to be 0.  The last column
contains the name of the outcome.  If no name is given, a default of
"Outcome \#" will be assigned.  When first started, the dialog will consist
of one blank row.  As the greatest row is filled in, a new empty row will
be created below it.  You can thus create any number of outcomes.  An
entry can be modified by moving the 'cursor' to it and typing in the
desired number.

An outcome is automatically saved/modified every time the cursor moves to
a different row.  Thus, you can use an outcome right after entering it and
moving the cursor the the next row.

To attach an outcome to a node, position the 'cursor' in the outcomes
dialog on any cell in the row that contains the desired outcome vector.
Position the cursor in the extensive form display window on the node an
outcome is to be assigned to.  Press the {\em Attach} button in the
outcomes dialog.  The extensive form display will update to reflect the
changes (assuming outcomes are being displayed).

\normalbox{For configuration and output features of this dialog see
the generic \helprefn{Table Window}{TableWindow} description.}

\subsection{Extensive Form Solutions}

All of the available solution algorithms are accessed through the
Solve->Solve menu.  This dialog provides for both the creation of new
solutions and inspection of already existing ones.  If a particular
algorithm is not aplicable for the current game, its selection will be
disabled.  If there already exist solutions created with the selected
algorithm, the {\em Look} button will become enabled.  Pressing this
button will take you to a solutions inspection window window.  That window
depends on the type of the algorithm that is selected.  For algorithms
that generate behavior strategy equilibria, the  \helpref{behavioral
strategy profile solutions}{ExtensiveSolutionInspect} window will appear.
No inspection windows exist for algorithms that generate pxi type files.

If the current game was generated from an normal form game, and if no
changes have been made to either game, there will exist a link between the
two forms.  In this case, the {\em Normal form} checkbox will be enables.
Checking this box will cause all generated solutions (if applicable to the
solution type) to be converted back into mixed profiles and projected back
to the normal form window.  The solutions can then be examined in the
normal form display.  {\tt THIS IS NOT YET IMPLEMENTED}

It is possible to use many of the normal form algorithms to solve
extensive form games by first converting them to normal form.  If one of
these algorithms is required, checking the {\em Use NF} checkbox will
enable the {\em Nfg Algorithms} choicebox.  A normal form representation
will be automatically created, the algorithm run and the solutions either
output to a file or converted back to behavioral strategy profiles
(depending on the algorithm type).

\subsubsection{Extensive Solutions Inspect}\label{ExtensiveSolutionInspect}
\begin{figure}
$$\image{3cm;3cm}{efgsoln}$$
\caption{Extensive Form Solution Inspect Window}\label{fig:nfgsoln}
\end{figure}

The solution inspection window is a very powerful and complex part of the GUI.
The basic functionality consists of displaying the BehaviorProfiles that contain the
probability data for the solutions.  Each player's infoset is displayed on a separate
line.  The {\em Iset} column gives the infoset ID in the form of (player \#,iset \#).
The next cell contains a vector of probabilities in which each entry corresponds to
an action that can be taken at that infoset.  If a player has more than one infoset,
additional lines are used.  This format is repeated for each player in the game.

The value of the solution to each player can be displayed by selecting the 
{\em Display Equilibrium Values} option.

The number of the solution that is currently selected will be hilighted in the first
column.  To change the current solution (and thus change the display in the 
corresponding EF window), double click (control-click in unix) on the \# of the
desired solution.  To quickly browse the solutions, the {\em Update Solutions Dynamically}
option may be used.  With this option on, the solution will change automatically once
the cursor moves to a different solution \#.  To remove solution display from the NF window,
double click on the first row.

A useful feature when navigating large games is {\em Hilight Infosets}.  It is enabled
from the EF window {\em Solve->Features}.  When enabled, double clicking on an
infoset in the EF window will hilight the corresponding infoset in the Solution inspect
window.  Double clicking on an infoset (Iset column) in the Solution inspection window
will hilight the corresponding infoset in the EF window.

If the underlying EF was generated from an NF, and if the link between the two is still
valid, the {\em EF->NF} button will be enabled.  To project the solution to the NF and
immediately display it there, press this button.  NOT IMPLEMENTED.
 

\subsection{Default Accelerator Keys}\label{ExtFormDefAccl}


\section{Table Window}\label{TableWindow}
The table window is used in many different placed throughout GAMBIT.  Thus
it offers a large number of possible configuration options.  The program
will attempt to choose the most suitable set of options for each use, but
the user is free to modify these.  All the modifications are accessed
through the Display->Options menu or the {\em Config} button, whichever is
present.

\begin{description}
\item[Label Font] Each row and column can be labeled.  This option controls
the size and the appearance of the font used for these labels.
\item[Data Font] This is perhaps the most useful option that allows the 
appearance of the actual data displayed in the table to be changed.  By
reducing the size of the font, more information can be made visible at the
same time.  Note that cell dimensions will usually change with the font.
\item[Cell Width] The program will usually select the cell width to fit the
widest item in the table.  However, if an unusually long name is used,
this width may be insufficient.  The scrollbar controls the width of the
cell, measured either in character width's or pixels.  Character based
sizing is default and is recommended.  If for some reason more precise
dimensioning is desired, or if the cell width is not to change with the
font size, the {\em char} checkbox should be unchecked.  The width can be
changed for a single column by choosing the desired column \# in the {\em
Col} choicebox or for all the columns by choosing All in the choicebox
(default).
\item[Show Labels] It is possible to turn off the row and column labels by 
unchecking their corresponding checkboxes.  However, this is not
recommended since frequently important information is contained in those
labels.
\item[Color Text] Some instances of the table window employ colored text 
for clearer presentation of the data (i.e. the normal form, solution
inspection windows).  The only reason to turn this feature off is to speed
up display on very slow computer or over very slow networks.
\end{description}

All of these features can be saved to a defaults file (gambit.ini).  They
will be taken into account the next time the table window is used, as long
as they are not overriden by the program.  Only the font information is
never overriden.


\section{Accelerator Keys}\label{Accelerators}

Many frequently executed commands are much more efficiently entered
through the keyboard than by using a mouse.  To speed up the use of the
GUI for an experienced use, most of the 'mouseable' commands can also be
done by entering a combination of keys on the keyboard.  Both the normal
and the extensive forms possess this functionality.  GAMBIT comes with a
pre-defined set of accelerator keys described in the corresponding
\helpref{extensive}{ExtFormDefAccl} and
\helpref{normal}{NormFormDefAccl} sections.  Any of these default command-key
associations can be edited or removed.  The accelerator key setting is
accessed through the {\em Display->Accel} menu.  In this dialog, the event
to be changed is first selected from the choicebox on the right.  If this
event is associated with a key combo already, the combo will be displayed
in the box on the left, otherwise, the entries in the box will be blank.
This association can be removed by pressing the {\em Delete} button.  A
new key combo can be assigned by selecting the combo in the box and
pressing {\em Set}.  If the desired key is not alphanumeric, its mneumonic
description can be selected from the choicebox, otherwise, it can just be
entered in the textbox.  A combination of Control \& Shift key modifiers
can be selected in the radioboxes below.  (Note that ??  means that either
state will be accepted).


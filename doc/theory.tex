\chapter{Game theory for GAMBIT}

This chapter describes some notation and concepts from game theory
that are necessary to understand how Gambit represents and solves
extensive form games.  This chapter assumes a basic understanding of game
theory.  Definitions of the main terms are given intuitively in the
Glossary.

\section{The \index{extensive form}{Extensive Form}}\label{extformsec}

The extensive form is a detailed description of a sequence of decisions 
that must be made by a group of individuals. To illustrate the basic ideas,
 Figure~\ref{fig:samp1} 
shows how the extensive form of a simple 
\popref{two player poker game}{pokergloss} (from \cite{mye:91} is represented
 in the Gambit GUI.  

\begin{figure}
$$\image{1cm;0cm}{poker_ef}$$
\caption{GAMBIT Display of Extensive Form of
		a Simple Two Player Poker Game}\label{fig:samp1}
\end{figure}

 An extensive form is a \popref{topological tree}{toptreegloss}, together with a
set of functions that attach labels to various parts of the tree: we need
to assign players to \popref{decision nodes}{decnodegloss} partition
decision nodes into \popref{information sets}{infosetgloss}, enumerate
the \popref{branches}{branchgloss}, assign probabilities to chance moves,
and attach \popref{outcomes}{outcomegloss} to nodes. In Gambit, the extensive
 form is drawn with the \popref{root node}{rootnodegloss} being the node 
furthest to the left,
branches going from left to right, and successive nodes to the right of
their predecessors.  So nodes further to the right represent decisions
that occur later in time.

In Gambit, a \popref{node}{nodegloss} is represented as a
horizontal line, whose color represents the player who moves at that node.
The length of two nodes or branches at the same level of the tree can be
different.  This is of no consequence.  The \popref{branches}{branchgloss} 
represent "actions, and are represented as composed of two connected 
segments, the ``\index{fork}{fork}'', and the
``\index{tine}{tine}''.  The fork is used in the graphics display to
indicate graphically the probabilities of taking each branch.  The tine is
used as a location at which to display textual information about the
branch.  In the graphics display that you see on your computer, the nodes
and branches are coded by the color of the player who controls the node.

\subsection{Information sets}
Nodes in an information set are called {\bf \index{members}{members}} of 
the information set.  Whenever possible, information sets are represented 
by vertical lines connecting
the \index{node tokens}{node tokens} of the nodes that are members of the 
same information
set.  See \helpref{Figure~\ref{fig:samp1}}{extformsec} for an example.  
It is only possible to do this when nodes
in the same information set are at the same level of the tree.  Usually
nodes in the same information set will be nodes that occur chronologically
at the same point in time.  But this is not always so.  In cases when two
nodes in the same information set are at different levels of the tree they 
will not be connected by a vertical line.  

\subsection{Labeling}
There are three locations at which each node can be labeled, (above, 
below, and to the right) and two at which each branch can be 
\index{label}{label} labeled (above and below.) You can choose to display
 any information you want in the various positions by selecting the 
information in the Display, Legend menu.  

\subsection{Numbering}\label{infosetnumbering}
In order  to associate strategy profiles in the
normal form with the corresponding behavioral strategies in the extensive
form, (and vice versa) you will need to understand how the branches and 
information sets are numbered.  The numbering of branches and information
 set is used to number strategies in the normal form.  

The \index{numbering}{numbering} of the \index{branches}{branches} is always
from top to bottom.  Thus, the highest branch at any node is always branch
1, the next branch is branch 2, etc. 

In Gambit, each \popref{information set}{infosetgloss} has a unique 
information set ID, consisting of the player followed by the information 
set number.  Information sets for each player are numbered consecutively, 
in \popref{indexed traversal order}{indextravgloss}.  
The information set ID can be displayed at the decision nodes by selecting 
it in the Options Legend menu.  

\section{\index{strategies}{Strategies} and The \index{normal form}{Normal Form}}

A \index{pure strategy}{pure strategy} for a player is a function that 
associates with each of
its information sets, a chosen branch at that information set.  A pure
strategy $n$-tuple is a vector of strategies, one for each player.  
The normal form for a game is a mapping which associates with each pure  
strategy $n$--tuple, an expected payoff to each player from 
playing that strategy. 

In the extensive form of the game in GAMBIT, the information sets for a 
player are numbered from $1$ to $J$, where $J$ is the number of information
 sets that player $i$ has.  We denote the pure strategy in which player $i$
 adopts strategy $a$ in its first information set, $b$ in its second, and 
$c$ in its third $abc$.  So if player $i$ has three information sets, where 
the first two information sets contain three branches and the last contains
 two branches, then player $i$ has a total of $18$ strategies, 
and ${\tt 312}$ indicates the strategy where the player chooses the third 
branch at its first information set, the first branch at its second 
information set, and the second branch at its third information set.  

In the extensive form of \helpref{Figure~\ref{fig:samp1}}{extformsec}, 
Player 1 has two
information sets, each with two choices, so it has a total of four 
strategies.  They are {\tt 11, 12, 21 } and {\tt 22}.  Player 2 has one
 information set, with two branches, so they are labeled {\tt 1} and {\tt 2}.
 The strategy {\tt 21} for player 1 represents the strategy of
choosing FOLD with a Red card, and RAISE with a Black card.  
 
\subsection{Wildcard notation}
ITo represent a collection of strategies which differ only in the choice at
a particular information set, we use a ``\index{wildcard}{wildcard}'' 
notation.  For example,
if a player has three choices at its second information set, then the 
notation {\tt 3*2} is used to represent the collection of strategies, 
{\tt \{312, 322, 332\}}.  

The {\tt *} notation in GAMBIT serves the same purpose as the ``wildcard'' 
character in DOS file specifications.  To get a good appreciation for how 
the wildcard works, you should exit GAMBIT, and WINDOWS, go to the root 
directory of your hard drive, and type \verb+del *.*[RET]+.

\subsection{\index{normal form}{Normal Form}}\label{normformsec}
The normal form of a game is a mapping which associates with 
each \popref{strategy profile}{stratprofgloss}, $s$, a vector of
 payoffs $u(s) = (u{s}(1), u{s}(2), \ldots , u{s}(n))$  for the $n$ players. 
 The normal form will thus be an $n$ dimensional matrix, with each cell 
containing a vector of length $n$.

Every extensive form game has an associated normal form.  The payoff for 
a given player under a particular strategy is computed as the sum of the 
\popref{realization probabilities}{realprobgloss} of each node times the 
value to the player of any outcome at that node.  The payoffs in the normal 
form are simply the expected payoff to each player from the given strategy
 profile.  

\begin{figure}
$$\image{1cm;0cm}{poker_nf}$$
\caption{GAMBIT Graphics Display of Normal Form of
		Simple Two Player Poker Game}\label{fig:samp2}
\end{figure}

Figure~\ref{fig:samp2}  gives the normal form for the extensive form game
 of poker illustrated in
\helpref{Figure~\ref{fig:samp1}}{extformsec}. 

In the game of  \helpref{Figure~\ref{fig:samp1}}{extformsec}, with the 
strategy profile {\tt (12, 1)}, the realization probability of the terminal
 node on the path RED, RAISE, MEET, with a payoff of $(\$2.00, -\$2.00)$
 is $1/2$, and the realization probability of terminal node on the path
 BLACK, FOLD, with a payoff of  $(-\$1.00, \$1.00)$ is $1/2$.    All other
 terminal nodes have realization probability of $0$ at this strategy profile.
 Taking expectied values, this gives a payoff of 
$(\$0.50, -\$0.50)$, which is the entry in the normal form for this cell.  

\section{Equilibria}

\subsection{\index{domination}{Domination}}\label{domsec}
Given two strategies, say $s, t$ for player $i$, we say that strategy $s$ 
dominates strategy $t$ if, in every \popref{contingency}{continggloss} $s$
 is at least as good as $t$, and there is some contingency in which it is 
better.  

In the poker example of \helpref{Figure~\ref{fig:samp1}}{normformsec}, 
for player 1, strategy \verb+22+ is dominated by strategy \verb+12+, 
and \verb+21+ is dominated by \verb+11+.  

\subsection{\index{Nash equilibrium}{Nash Equilibrium}}\label{nashsec}
A Nash equilibrium is a strategy profile having the property that no player
 can strictly benefit from unilaterally changing its strategy, while all 
other players stay fixed.  

Every finite game has at least one Nash equilibrium in either pure or 
mixed strategies (\cite{Nash:1950}).  GAMBIT can find all pure strategy 
Nash equilibria for the game, and if there are no pure strategy equilibria, 
GAMBIT will find at least one mixed strategy Nash equilibrium.  For a two 
person game, it can find all Nash equilibria, pure and mixed, if the game 
is sufficiently small.  

\subsubsection{Pure Nash Equilibria}\label{purenashsec}
To find pure strategy Nash equilibria, for a normal form game, in the Solve 
menu select Solve EnumPure.  To find pure Nash equilibria for an extensive
 form game,  in the Solve menu, select EnumPure, and Use NF.  This will find 
all pure strategy nqsh equilibria of the associated reduced normal form of 
the game, and convert them to behavoioral strategies.  

\subsubsection{Mixed Nash Equilibria}\label{mixednashsec}

\subsection{\index{Sequential Equilibrium}{sequential equilibrium}}\label{seqnashsec}
Sequential equilibria are equilibria that prescribe optimal behavior at any 
information set of the extensive form, given a consistent assessment of 
beliefs.  See \cite{KrepsWilson:1982}.  To compute an approximation to a
 sequential equilibrium you can select LqreSolve in the Extensive form Solve
 menu. 

\section{Types of Games}
\subsection{\index{Perfect Recall}{perfect recall}}\label{perfrecallsec}
A game of perfect recall is a game in which individuals do not forget what
they have known or done earlier in the game.  This requires that
information sets must refine earlier information sets, and if there is
an information set, one of whose nodes follows a choice by the same player
at a previous information set, then all of the members of that information
set must follow the same choice.

GAMBIT does not enforce perfect recall on the extensive form games that
you build.  In fact, GAMBIT will solve games without perfect recall for
 optimal
mixed strategies.  The problem comes in converting the mixed strategy back
to a behavioral strategy.  If the game has perfect recall, then by Kuhn's
theorem (see eg., \cite{vanDamme:1983}), any mixed strategy can be
converted back to a behavioral strategy which is realization equivalent.  
If the game does not have perfect recall, this is not always possible.  If
a game without perfect recall is solved, and there is no realization
equivalent behavioral strategy, you will be warned that there is a
problem. [Note -- this option is not yet implemented].  
 

\subsection{Games of \index{Incomplete Information}{incomplete information}}\label{incinfsec}
Games of incomplete information are games in which different players have 
different beliefs about the underlying parameters (such as the utility 
functions, strategy sets, or number of players) of the game.  The standard 
way of treating such games is that of \cite{Harsanyi:1967}, who showed that
 such games were equivalent to games in which players have some common
prior distribution over the joint distribution of characteristics, and
 then individuals observe their own type, but not that of the other players 
in the game.  

Games of incomplete information can be modelled in GAMBIT having an initial 
chance move determine the distribution of types, and then have each 
individual observe its own type.    



\documentstyle[a4,makeindex,verbatim,texhelp,fancyhea,mysober,mytitle]{report}%
\input psbox.tex
%\newcommand{\bd}{\begin{description}}
%\newcommand{\ed}{\end{description}}
\parskip=10pt%
\parindent=0pt%
\itemsep{0pt}
\helpfontsize{11}
\title{Gambit Graphics User Interface\\$$\image{1cm;0cm}{gambit}$$\\
\centerline{An Interactive Extensive Form Game Program}}% 
\author{Developed by: Richard D. McKelvey\\
Main Programmer: Theodore Turocy\\
Front End: Eugene Grayver\\
California Institute of Technology\\ \today.\\
Part of the Gambit Project:\\
Richard D. McKelvey and Andrew McLennan, PI's\\
Funding provided by the National Science Foundation}

\makeindex%
\begin{document}%
\maketitle%

%\pagestyle{fancyplain}
\bibliographystyle{plain}
%\pagenumbering{roman}
%\setheader{{\it CONTENTS}}{}{}{}{}{{\it CONTENTS}}
%\setfooter{\thepage}{}{}{}{}{\thepage}
\tableofcontents%

\chapter*{Copyright notice}%
%\setheader{{\it COPYRIGHT}}{}{}{}{}{{\it COPYRIGHT}}%
%\setfooter{\thepage}{}{}{}{}{\thepage}

Copyright (c) 1995, The Gambit Project, at California Institute of
Technology and University of Minnesota.  

The Gambit Project is funded in part by National Science Foundation grants
SBR-9308637 to the California Institute of Technology and SBR-9308862 to
the University of Minnesota.  

Permission to use, copy, modify, and distribute this software and its
documentation for any purpose is hereby granted without fee, provided that the
above copyright notice, author statement and this permission notice appear in
all copies of this software and related documentation.

THE SOFTWARE IS PROVIDED ``AS-IS'' AND WITHOUT WARRANTY OF ANY KIND, EXPRESS, IMPLIED OR OTHERWISE, INCLUDING WITHOUT LIMITATION, ANY WARRANTY OF MERCHANTABILITY OR FITNESS FOR A PARTICULAR PURPOSE.

IN NO EVENT SHALL THE GAMBIT PROJECT, THE CALIFORNIA INSTITUTE OF TECHNOLOGY, THE UNIVERSITY OF MINNESOTA, OR ANYONE ASSOCIATED WITH THE DEVELOPMENT OF COMPUTER SOFTWARE UNDER THE GAMBIT PROJECT, BE LIABLE FOR ANY SPECIAL, INCIDENTAL, INDIRECT OR CONSEQUENTIAL DAMAGES OF ANY KIND, OR ANY DAMAGES WHATSOEVER RESULTING FROM LOSS OF USE, DATA OR PROFITS, WHETHER OR NOT ADVISED OF THE POSSIBILITY OF DAMAGE, AND ON ANY THEORY OF LIABILITY, ARISING OUT OF OR IN CONNECTION WITH THE USE OR PERFORMANCE OF THIS SOFTWARE.


\chapter*{Acknowledgements}%
%\setheader{{\it ACKNOWLEDGEMENTS}}{}{}{}{}{{\it ACKNOWLEDGEMENTS}}%
%\setfooter{\thepage}{}{}{}{}{\thepage}

The Gambit Project is a project for the development of computer code for the 
solution of extensive and normal form games.  The software developed under 
the project is public domain.  The Gambit Project is funded in part by 
National Science Foundation grants SBR-9308637 to the California Institute 
of Technology and SBR-9308862 to the University of Minnesota.  

Numerous students at Caltech and the University of Minnesota have contributed
 to the Gambit Project:  Among these are Bruce Bell,  Anand Chelian, Nelson 
Escobar, Todd Kaplan, Brian Trotter, and Gary Wu. 

\chapter{Introduction} Gambit is a library of computer programs, written
in \verb$C++$, for building, analysing, and solving n-person games, in
eiither extensive or normal form.

The Gambit Graphics User Interface (\popref{Gambit GUI}{guigloss}\) is an
interactive, menu driven program for accessing this library of programs.
The Gambit GUI allows for the interactive building and solving of
extensive and normal form games.  It consists of two separate modules, one
for extensive and the other for normal form games.   In each module, you
can build, save load and solve a game, and convert back and forth from one
form of game to the other.

Despite it's ease of use, the Gambit GUI is not suitable for repetitive or
computer intensive  operations.  A separate program, the Gambit Command
Language (\popref{GCL}{gclgloss}) is designed to be used for such operations 
on games.  The
GCL is a program which allows access to the functionality of the gambit
llibrary of programs through a high level programming language.  The
Gambit GUI and GCL  are compatible with each other, in the sense that they
each generate files that can be read by the other, and they call the same
underlying library of functions.


\chapter{Hardware and Installation}
\section{Platforms}
The current version of gambit \popref{GUI}{guigloss}\ relies on the wxWin 
library.  The following platforms are currently supported by wxWin:
\begin{itemize}
\item IBM PC and compatible.  Gambit will run on any IBM PC compatible machine
equiped with MS Windows 3.1 and above.  Versions exist to support machines
with and without a math co-processor.  A port to Win32 is in the works.
\item Sun workstations.  Gambit is supported on most Sun workstations running
either SUN-OS or Solaris.  Either MOTIF or XView toolkits are required.
\item IBM RS/6000. Gambit is supported on IBM RS/6000 machines running motif.
\item Macintosh. A Macintosh version of the wxWin library is in the works.  As
soon as it becomes available, GAMBIT will be ported to that platform as well.
\item OS/2.  An OS/2 version of the wxWin library is in the works.  As
soon as it becomes available, GAMBIT will be ported to that platform as well.
Until then, Gambit GUI will run in a windows box under OS/2.
\end{itemize}
\section{Installation}
All of the gambit files can be found at the Gambit World Wide Web site
at \verb+http://hss.caltech.edu/~gambit/Gambit.html+

\begin{description}
\item[Unix] 
First of all make sure to get the correct version for your machine and
windows toolkit.  If unsure, contact your system administrator.  Follow
the instructions above for getting the file(s).  Type {\tt tar -xvf
[filename]}, where [filename] is the name of the file you just got.  This
will create a directory Gambit and place all the required files into it.
If you wish to use the PXI program, make sure to get the files required
for it, and follow the same steps.  You can now run the program by typing
{\tt gambit}.
\item[Windows] 
If you obtained the program through ftp, place the zip
file into a temporary directory and execute it by typing {\tt gambzip}.  If you
obtained the program on a floppy disk, just place the floppy into your drive.
Now execute the program {\tt setup} in the current location of gambit files.
Setup will automatically copy all the needed files to your drive, and create
an icon in your program manager.
\end{description}


\input theory.tex

\input algs.tex

\input gui.tex %-------------------------------- GUI --------

\bibliography{gambit}
\addcontentsline{toc}{chapter}{Bibliography}
\setheader{{\it REFERENCES}}{}{}{}{}{{\it REFERENCES}}%
\setfooter{\thepage}{}{}{}{}{\thepage}%

\begin{helpglossary}
\setheader{{\it GLOSSARY}}{}{}{}{}{{\it GLOSSARY}}%
\setfooter{\thepage}{}{}{}{}{\thepage}%

\gloss{Gambit GUI}\label{guigloss}
This Program -- the Gambit Graphics User Interface.

\gloss{GCL}\label{gclgloss}
The Gambit Control Language -- a programming language for manipulation and
solution of extensive and normal form games.  This language is designed
for repetetive and computer intensive operations on games.  
 

\gloss{Topological Tree}\label{toptreegloss}
A topological tree (also referred to as a Game tree) is simply a
collection of nodes which are connected together by branches in a way that
looks like a tree.  The Game tree represents a sequence of choices in the
chronological order that they occur.

\gloss{Node}\label{nodegloss}
A node is either decision node or a terminal node.  In the graphics 
display, a node is any location to which you can navigate the graphics 
cursor. In Figure~\ref{fig:samp1}, there are a total of eleven nodes -- namely five decision nodes (including the chance node), and six terminal nodes.  

\gloss{Decision Node}\label{decnodegloss}
A decision node is a node which has at least one node that follows it.  
Decision nodes represent points at which either a player or chance must 
make a decision.  The different choices are represented by branches.

In Figure~\ref{fig:samp1}, there are a total of five decision nodes (including the chance node).  The number of the player who makes a choice at a node appears underneath the
node as the first number after the parenthesis.  On the graphics screen,
decision nodes are color coded, with each player represented by a
different color.

\gloss{Terminal Node}\label{termnodegloss}
A terminal node is a node which has no other node following it.  Terminal 
nodes represent points at which the extensive form game ends, and typically 
have outcomes attached to them. 

In Figure~\ref{fig:samp1}, there are six terminal nodes, each followed by
a pair of numbers representing the payoffs to each player when that
terminal node is reached.

\gloss{Root Node}\label{rootnodegloss}
The root node of a tree is the node with no parent. In the GAMBIT graphics
representation, the Root node is always the node that is furthest to the
left.  It can be reached by successively pressing {\bf $\leftarrow$}. 

In Figure~\ref{fig:samp1}, the Root node is the unlabeled node, furthest to 
the left.

\gloss{Branch}\label{branchgloss}
A branch is the line connecting two nodes, one of which immediately 
follows the other.  Branches are numbered from 1 to k, where k is the 
number of branches at the node.  Branch 1 is the uppermost branch.  Branch 
2 is the second uppermost branch, etc. 

In Figure~\ref{fig:samp1}, each decision node has two branches, but in
other extensive forms, the number of branches can be different.

\gloss{Information Set}\label{infosetgloss}
An information set is a collection of nodes which are all controlled by
the same player, but which are indistinguishable 
for the player at the point it is making a decision.  Since any two nodes 
in the same information set are indistinguishable, they must have exactly 
the same number of immediately following nodes. 

In Figure~\ref{fig:samp1}, player 1 has two information
sets, labeled (1, 1) and (1, 2), and player 2 has one, labeled (2, 1).
Player 2 has only one information set because Player 2 does not know whether
 Player 1 drew a red card or a black card from the deck.  So 2's decision 
cannot be contingent on that information.

\gloss{Outcome}\label{outcomegloss}
Outcomes are payoff vectors that associate with each player a payoff.  
Outcomes can be attached to any node, terminal or non terminal, in the 
extensive form of the game.  When play passes or terminates at a node 
with an outcome attached to it, each player accumulates the payoff which 
is associated to that player by the outcome at that node. 

In Figure~\ref{fig:samp1}, there is an outcome attached to each of the
terminal nodes.  They are indicated by the pair of numbers at each
terminal node.  The first number represents the payoff to player 1 and the
second the payoff to player 2.

\gloss{Subtree}\label{subtreegloss}
A subtree is a decision node together with the collection of nodes that
follow it in the tree.  Note that any subtree itself has a tree structure.

In Figure~\ref{fig:samp1}, each decision node defines a subtree consisting of itself
and its followers.   

\gloss{Subgame}\label{subgamegloss}
A subgame is a subtree with the property that for every information 
set containing members in the subtree, all members of the information 
set are elements of the subtree.  In other words, every information set 
is either contained in or has empty intersection with the subtree. 

In Figure~\ref{fig:samp1}, there are no proper subtrees of the extensive
form, because every subtree except that starting at the Root node breaks
up Player 2's information set.

\gloss{Indexed Traversal Order}\label{indextravgloss}
This is the ordering imposed on the nodes of a game tree by a lexicographic 
ordering of the nodes when each node is identified by the sequence of branch 
numbers necessary to reach it.  


\gloss{Poker Description}\label{pokergloss}
\begin{verbatim} 
Each player ante's $1.00 into the pot before starting
Player 1 (RED) draws a card from a deck. Player 1 observes the card, 2 (BLUE) does not. 
Player 1 then decides whether to FOLD or RAISE. 
  If player 1 chooses FOLD, player 2 wins the pot (a net loss of $1.00 to RED).
  If player 1 chooses RAISE, then player 1 throws a dollar in the pot, and player 2 has a move
    If player 2 chooses PASS, then player 1 wins the pot (a net gain of $2.00to RED)
    If player 2 chooses MEET, player 2 throws a dollar in the pot, and 1 must show the card: 
      If the card is RED Player 1 wins the pot (a net gain of $2.00 to RED).
      If the card is BLACK Player 2 wins the pot (a net loss of $2.00 to RED)
\end{verbatim}

\gloss{PureStrategies}\label{purestratgloss} 
A pure strategy for player i is a plan of action for that player for the 
entire game.  Thus, it is a specification of what branch to select at each 
of the player's information sets.  If player i's j th information set has 
k(j) branches.  Then the total number of pure strategies for player i is 
k(1) x k(2) x . . . x k(J).

\gloss{Strategy Profile}\label{stratprofgloss}
A strategy profile is an n-tuple of strategies, one for each player. The 
total number of strategy profiles is equal to the product of the number of 
strategies for each player. 
 
In the game of  Figure~\ref{fig:samp1}, the total number of strategy profiles
 is 8. 

\gloss{Realization Probability}\label{realprobgloss}
With each strategy profile, there is associated, to each node in the 
extensive form, a realization probability.  This is the probability of
 reaching that node under the given strategy profile.  This probability is 
computed by finding the path from the root node to the given node, and then 
computing the product of the probabilities of selecting each branch along 
the path.  

Note that all nodes, not just terminal nodes have realization probabilities
 attached to them.  The realization probability of the decision node at RED 
RAISE is 0.5.  The realization probability of the Root node is always 1.

\gloss{Contingency}\label{continggloss}
A contingency for player i is a profile of strategies adopted by the other
 n - 1 players.

\gloss{Reduced Normal Form}{rednormgloss}
The reduced normal form is the game that results when all strategies that 
are identical (result in  the same payoffs for all contingencies of the 
other players are represented by a single strategy.  

In many extensive form games, there are cases in which a choice adopted by 
a player early in the play precludes that player from ever reaching other 
of its information sets.  In this situation, the decisions that the player 
makes at the unreached information sets can not possibly affect the  payoff
 to that player(or to any other player).  Two strategies for a given player
 that differ only in what branch is selected at an unreachable'  information
 set will generate rows (or columns, as the case may be) in the normal form 
that are identical.  The reduced normal form eliminates such duplicated 
strategies, representing each duplicated strategy by its wildcard notation.  



\end{helpglossary} %

\end{document}


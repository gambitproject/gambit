%
% $Header$
%
% Description: LaTeX source for GCL manual
%

\documentclass[12pt]{report}
\usepackage{../texhelp}
\usepackage{verbatim}
\usepackage{makeidx}
\usepackage{longtable}
\renewcommand{\baselinestretch}{.9}
\input{../psfig}

\newcommand{\bindex}[1]{{\bf #1}\index{#1}}
\newcommand{\rindex}[1]{{#1}\index{#1}}

% single-space the two-column environment
\twocolspacing{1}

\newcommand{\foralltypes}{\noindent \textit{For all types} \texttt{T}}
\renewcommand{\description}[1]{\vspace{6pt}\noindent #1}
\newcommand{\note}{\vspace{6pt}\noindent\textbf{Note:} }
\newcommand{\seealso}{\vspace{6pt}\noindent\textbf{See also:} }
\newcommand{\shortform}{\vspace{6pt}\noindent\textbf{Short form:} }
\newcommand{\udfbody}{\vspace{6pt}\noindent\textbf{Definition:} }
\newcommand{\funcxref}[2]{\helpref{\texttt{#1}}{#2}}
\newcommand{\udfxref}[2]{\helpref{\texttt{#1} (user-defined)}{#2}}
\newcommand{\primitivefunction}[1]{\section*{#1}\label{Prim#1}\index{#1 (GCL primitive function)}}
\newcommand{\newsignature}{\htmlignore{\noindent\rule{2in}{0.5pt}}\latexignore{\hrule}\noindent}

%\renewcommand{\popref}[2]{#1}

% stuff for lex 
% to put dictionary style headers in function ref sections

\newcommand{\lex}[1]{\mark{#1}\index{{\tt #1}}}

%\makeatletter
%\def\ps@lexicon{\def\@oddhead{\slshape\mysectionname\hfil 
%{\Large\verb+\botmark+}\quad\thepage}
%\def\@evenhead{\thepage\quad{\Large\verb+\firstmark+}\hfil \slshape\mysectionname}}
%\makeatother

\newcommand{\mysectionname}{\thesection\,\, BUILT-IN FUNCTIONS}

% end stuff for lex

\latexonly{\oddsidemargin = 0in}
\latexonly{\evensidemargin = 0in}
\latexonly{\textwidth = 6.5in}

\pagestyle{headings}

\bibliographystyle{plain}

\title{Gambit Command Language\thanks{This is part of
the Gambit Project, which was funded in part by National Science
Foundation grants SBR-9308637 and SBR-9617854 to the California
Institute of Technology and SBR-9308862 to the University of
Minnesota.  We are also grateful to the Intel Corporation for computer
equipment grants to California Institute of Technology that
facilitated this project}\\Version 0.97}

\author{Richard D. McKelvey \\
California Institute of Technology \\
\htmlignore{\and}
Andrew McLennan \\
University of Minnesota \\
\htmlignore{\and}
Theodore Turocy \\
Texas A\&M University}

\date{\today}

\makeindex
\begin{document}
\maketitle

\tableofcontents

\chapter*{Copyright Notice}

Copyright (c) 1994-2001, The Gambit Project, at California Institute of
Technology and University of Minnesota.

Permission to use, copy, modify, and distribute this software and its
documentation for any purpose is hereby granted without fee, provided
that the above copyright notice and this permission notice appear in
all copies of this software and related documentation.

THE SOFTWARE IS PROVIDED ``AS-IS'' AND WITHOUT WARRANTY OF ANY KIND,
EXPRESS, IMPLIED OR OTHERWISE, INCLUDING WITHOUT LIMITATION, ANY
WARRANTY OF MERCHANTABILITY OR FITNESS FOR A PARTICULAR PURPOSE.
 
IN NO EVENT SHALL THE GAMBIT PROJECT, THE CALIFORNIA INSTITUTE OF
TECHNOLOGY, THE UNIVERSITY OF MINNESOTA, OR ANYONE ASSOCIATED WITH THE
DEVELOPMENT OF COMPUTER SOFTWARE UNDER THE GAMBIT PROJECT, BE LIABLE
FOR ANY SPECIAL, INCIDENTAL, INDIRECT OR CONSEQUENTIAL DAMAGES OF ANY
KIND, OR ANY DAMAGES WHATSOEVER RESULTING FROM LOSS OF USE, DATA OR
PROFITS, WHETHER OR NOT ADVISED OF THE POSSIBILITY OF DAMAGE, AND ON
ANY THEORY OF LIABILITY, ARISING OUT OF OR IN CONNECTION WITH THE USE
OR PERFORMANCE OF THIS SOFTWARE.

\chapter*{Acknowledgements}

The Gambit Project is a project for the development of computer code
for the solution of extensive and normal form games.  The software
developed under the project is public domain. The Gambit Project is
funded in part by National Science Foundation grants SBR-9308637 and
SBR-9617854 to the California Institute of Technology and SBR-9308862
to the University of Minnesota.

The Gambit Project is an ongoing project, where the set of active
participants has changed over time, and will be expected to change
further in the future.  The main contributors to this release are
Richard D. McKelvey, Andrew McLennan and Ted L. Turocy.  However, this
release builds heavily on the previous releases, and we would be
remiss not to acknowledge the significant contributions of those
involved in those previous releases.

The Gambit project was begun in the mid 80's, at which time it
consisted of an implementaion of the GUI in Basic.  The code has since
been totally rewritten twice (first in C, then in C++), the
functionality has been expanded, and the GCL has been developed as an
alternative to the GUI to support econometric and computationally
intensive use.

Richard McKelvey has directed the Gambit project since its inception
in the mid 80's, and was one of the principal investigators on the NSF
grants for the Gambit Project.  He designed the original GUI, was
involved in the design of the GCL (1993), and has been responsible for
implementation of many of the algorithms in gambit.

Andrew McLennan has worked on the project since 1993, and was one of
the principal investigators on the NSF grants for the Gambit project.
Andy was involved in the original design of the GCL (1993).  He has
been responsible for the implementation of the polynomial solution
code, which finds all roots of polynomial systems of equations, and
(effective release 0.96) is used as the default algorithm to find all
Nash equilibria of n-person games, when n is greater than two.

Ted Turocy has worked on the project since 1992, working full time for
a year and a half of that time.  He has been the main programmer for
the project, also responsible for supervising other programmers
working on the project.  Ted converted the code from C to C++, changed
it to an object oriented style, wrote code for the container classes
and for the current implementaton of the extensive and normal form
classes, and implemented the GCL.  He also made substantial
improvements to the GUI for this release.

In listing the contributors to previous releases, we must
first mention Eugene Grayver and Gary Wu, each of whom worked
for at least four years on the project, making major contributions:

Eugene Grayver worked on the project from 1994 through 1997.
He wrote the code for the wxwin implementation of the GUI for versions
.92 through .94, developed our Web page, handled the software
distributions, and implemented the console interface for version .94
of the GCL.

Gary Wu worked from 1995 through 1998 on the project.  He was
responsible for large portions of the GCL implementation, including
the implementation of ``listability'', implementation of many of the
built in functions, the command line editor, the original stack
machine, and the Windows MFC interface for the GCL.

Numerous additional students at Caltech and the University of Minnesota
have contributed to the Gambit Project:

Bruce Bell worked in 1989 on the first $C$ version of the Gambit GUI,
which was distributed in 1991-92 as version .13, and later (1994-1995)
helped with the implementation of the tableau and LU decomposition
code (used in the two person LCP and LP algorithms). Matthew Derer
worked in the summer of 1993 to do an initial implementation of the
GUI for XWindows.  Anand Chelian worked summer of 1994 to do an
initial implementation of the matrix and vector classes.  Brian
Trotter worked summer of 1994 on internals of the container classes
and the extensive form classes.  Todd Kaplan worked in the summer of
1994 and did initial work towards implementaion of a multivariate
polynomial class.  Nelson Escobar worked over the summer of 1995 to
improve the LUDecomposition code, and worked on a second
implementation of the polynomial classes.  Rob Weber did extensive
testing from 1994 to 1996, and helped with the documentation of
version .94.  Geoff Matters worked summer of 1997 to make various
improvements to the LP solver and on improvements to the built in
function definitions in the GCL.  Ben Freeman worked in the summer of
1998 to implement the integer tableau class, which is used to speed up
the rational computations for two person games.  Michael Vanier worked
summer of 1998 through the spring of 1999 to set up the CVS repository
system for source code control, work on the distribution scripts, and
make various bug fixes to the GUI.

We are also indebted to Bernhard von Stengel at the London School of
Economics for help and advice on implementation of the sequence form
and for contributing his ``clique'' code for identification of
equilibrium components in two person games.

\input introduction.tex
\input concepts.tex
\input advanced.tex
\input games.tex
\input solving.tex

\input funccat.tex
\input funcalpha.tex
\input funcudf.tex

\bibliography{gambit}

\helpignore{\input{gclman-html.ind}}

\end{document}





